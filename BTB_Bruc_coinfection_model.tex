\documentclass[letterpaper,12pt]{article}
\usepackage{booktabs,graphicx,amssymb,lineno,amsmath,multirow,rotating,verbatim,setspace,float,subfigure, fixltx2e}
\usepackage[margin=1in]{geometry}
\usepackage{gensymb}
\usepackage[table]{xcolor}

%*******************************************************************************************************
% verbatim lets you do \begin{} and \end {} comment
%amsymb & amssymb useful for math.  
% check out the wiki page 
%/cite  or citep (with natbib package?)
%/textit or 

%added fixlt2e 

% Latex once to get the file
% Then bibtex (Put in the same WD)
% then typeset LaTex again by pushing typeset twice to get the citations in 

\begin{document}

\linenumbers
%\renewcommand{\thesubfigure}{\Alph{subfigure}}
%\bibpunct{[}{]}{,}{n}{,}{,}

\title{Summary of BTB-brucellosis co-infection model and assumptions}
\date{23-April-2016}

\maketitle
\doublespacing

\begin{figure}
\begin{center}
\includegraphics[width=4in]{Figure1_ageprev.pdf}
\end{center}
\caption{Age, brucellosis-prevalence curves in buffalo simultaneously uninfected (blue) or infected (red) with bTB.  The inset figure shows the distribution of ages in our sample, which best cover buffalo between two and eight years old and include 796 samples from 151 repeatedly captured individuals.  Ages are binned for visualization and animals that test positive are assumed to remain positive for the duration of the study.  Sample sizes for TB- are: 10, 182, 209, 103, 27, 35 buffalo.  Sample sizes for TB+ animals are: 0, 21, 108, 75, 21, 1 for ages 0-2, 2-4, 4-6, 6-8, 8-10, and 10+ respectively.}
\label{fig1}
\end{figure}

We observed a positive association between BTB and brucellosis in our field data.  
Longitudinal observation showed that this pattern was age-specific, with the positive association occurring only in younger buffalo.  
We developed a model for BTB-brucellosis co-infection to understand the mechanisms and consequences of co-infection.
Parameter values representing the consequences of co-infeciton in this model are estimated from field data.  
We and use the model to ask, (1) are the observed changes in transmission and mortality rates sufficient to generate the positive co-infection patterns observed; (2) what is the consequence of brucellosis infection for the invasion of bTB; and (3) what is the consequence of bTB invasion for the dynamics of brucellosis (endemic equilibrium values)?

\pagebreak


\section*{Model Structure and Assumptions}

We developed a continuous time differential equation model to evaluate the disease dynamic consequence of bTB invasion on brucellosis dynamics and vice versa.  
Our model structure reflects the costs of co-infection identified for each parameter: host mortality rate, host birth rate, and disease transmission rates (Figure 2).
Animals are represented with six groups: susceptible to both infections($S$), acutely infected with brucellosis ($I_{B}$), chronically infected with or recovered from acute brucellosis ($R_{B}$), infected with bTB ($I_{T}$), co-infected with both pathogens ($I_{C}$), or in the chronic stages of brucellosis and infected with bTB ($R_{C}$). 

\begin{figure}
\begin{center}
\includegraphics[width=6in]{Figure1_PNAS.jpg}
\end{center}
\caption{(center) Schematic representation of the disease model defined in SI Appendix 3.  Hosts are represented as Susceptible (S), infected with TB only (IT), infected with brucellosis only (IB), co-infected with both infections (IC), persistently infected with brucellosis only but no longer infectious (RB), and persistently infected with brucellosis and co-infected with BTB (RC). (left) A detailed cohort study informs model parameterization, including the mortality, transmission, and fecundity consequences of co-infection as well as the (right) transmission parameters for both infections.  The prevalence plot illustrates that the model accurately reproduces the observed co-infection patterns in the data.  Bars represent model results and dots represent the data.}
\label{fig2}
\end{figure}


% In document without aging in...
\begin{align*}
\frac{dS(a)}{dt} &= r(a, N) - \beta_{T} T S(a) - \frac{\beta_B (a) B}{N} S(a) - \mu_{S}(a) S(a) \\         
\frac{dI_{T}(a)}{dt}&= \beta_T T S(a) -  \frac{\beta'_{B}(a) B}{N} I_{T}(a) - \mu_{T}(a) I_{T}(a) \\
\frac{dI_{B}(a)}{dt}&=  \frac{\beta_B (a) B}{N}S(a) - \beta'_{T} T I_{B}(a) + \epsilon R_{B}(a)  - (\gamma + \mu_{B}(a)) I_{B}(a) \\
\frac{dI_{C}(a)}{dt}&=  \beta'_{T} T I_{B}(a) + \frac{\beta'_{B}(a) B}{N} I_{T}(a) + \epsilon R_{C}(a)  - (\gamma + \mu_{C}(a)) I_{C}(a)\\  
\frac{dR_{B}(a)}{dt}&=  \gamma I_{B}(a) - (\epsilon + \mu_{B}(a)) R_{B}(a) \\            
\frac{dR_{C}(a)}{dt}&=  \beta'_{T} T R_{B}(a) + \gamma I_{C}(a) - (\epsilon + \mu_{C}(a)) R_{C}(a) \\ 
\end{align*}


\textbf{Parameter table - not yet updated}
\begin{table}[H] %hb
\newcommand{\head}[1]{\textnormal{\textbf{#1}}}
\small
\begin{tabular}{lcp{12cm}}
\hline
\head{ } & \head{Value} & \head{Meaning}\\
\hline
$b$ & $0.54$ &  maximum natural birth rate for uninfected adults  \\
b1 & $1$ &  proportional reduction in births for adults with bTB \\
b2 & $1$ & proportional reduction in births for adults with brucellosis.  \\
b3 & $1$ & proportional reduction in births with chronic brucellosis.  \\
b4 & $1$ & proportional reduction in births for co-infected adults. \\
b5 & assume= b4 & proportional reduction in births for co-infecteds with chronic brucellosis.  \\
$\mu_{S_i} $ & $ 0.12, 0.04$ & annual mortality rate for uninfected juveniles and adults\\
$\mu_{T_i} $& $2.82 \mu_{S_i} $ & annual mortality rate for animals with bTB \\
$\mu_{B_i} $& $3.02 \mu_{S_i} $ & annual mortality rate for animals with brucellosis \\ 
$\mu_{R_{B_i}} $& $\mu_{B_i}$ & annual mortality rate for animals with chronic brucellosis \\ 
$\mu_{C_i}$ & $5.84 \mu_{S_i} $ & annual mortality rate for animals with bTB-brucellosis co-infection \\ 
$\mu_{R_{C_i}}$ &  $\mu_{C_i}$ & annual mortality rate for co-infecteds with chronic brucellosis \\ 
%K& $1000$ & indiv & carrying capacity \\
$\beta_T $ & fit to data & transmission rate for bTB in susceptible animals \\
$\beta_B $ & fit to data & transmission rate for brucellosis in susceptible animals \\
$\beta_{T}^{'}$ & $\beta_T$ & transmission rate for the bTB in animals infected with brucellosis \\
$\beta_{B}^{'}$ & $2.1 \beta_B$ & transmission rate for brucellosis in animals infected with bTB \\
$\gamma$& $ 1/2 $& 1/duration of infectious period for brucellosis \\
$\epsilon$& $0.01$ & recrudescence rate \\
$\rho$& $0.05$ & proportion of births from animals with brucellosis that result in infection \\
\hline 
\end{tabular}
\end{table}
\caption{Parameter table- not yet updated}
\\

\pagebreak








WORK STOPS HERE




\pagebreak 

%%%%%%%%%%%%%%%%%%%%%%%%%%%%%%%%%%%%
% Model analysis without aging
\begin{align*}
N & = S + I_{T} + I_{B} + R_{B} + I_{C} + R_{C} \\
T &= I_{T} + I_{C} + R_{C} \\
B &= I_B+I_C \\
N_{b} &= S + b_1 I_{T}+ b_2 I_{B} + b_3 R_{B} + b_4 I_{C} + b_5 R_{C} \\
\frac{dS}{dt} &= b N_b \big(1 - \frac{r}{b} \frac{N}{K}\big) - \beta_T T S - \beta_B B S - \mu_{S} S  \\		      			% Susceptibles 
\frac{dI_{T}}{dt}&= \beta_T T S -  \beta'_{B} B I_{T} - \mu_{T} I_{T} \\									% TB 
\frac{dI_{B}}{dt}&=  \beta_B B S - \beta'_T T I_{B} - \gamma I_{B} + \epsilon R_{B}  - \mu_{B} I_{B} \\ 		 	% Brucellosis
\frac{dI_{C}}{dt}&=  \beta'_T T I_{B} + \beta'_{B} B I_{T} - \gamma I_{C} + \epsilon R_{C}  - \mu_{C} I_{C}  \\  		% Co-infecteds 
\frac{dR_{B}}{dt}&=  \gamma I_{B} - \epsilon R_{B} - \mu_{RB} R_{B} \\  												% Chronic brucellosis 
\frac{dR_{C}}{dt}&=  \beta'_T T R_{B} + \gamma I_{C} - \epsilon R_{C} - \mu_{R_{C}} R_{C} \\ 						% Chronic Brucellosis and co-infecteds
\end{align*}

where $N$ is the total host population and T, B, and $N_b$ are the numbers of buffalo contributing to tuberculosis transmission, brucellosis transmission, and births respectively. \\

To evaluate the consequences of co-infection for disease invasion, we evaluate how co-infection alters the basic reproduction number ($R_{o}^{T}$) of bTB. \\
Lets first consider the bTB only model: 
\begin{align*}
\frac{dS}{dt}&=b (S + b_1 I_T)\left(1-\frac{r}{b}\frac{S+I_T}{K}\right) -\beta_T S I_T - \mu S\\
\frac{dI_T}{dt}&=\beta_T S I_T - \mu_{T} I_T
\end{align*}

$R_{o}^{T}$ for bTB in the case when brucellosis is absent is, $R_{o}^{T} = \frac{\beta_T K}{\mu_T}$.  
The disease free equilibrium is $E_o = (K, 0)$, where the variables are denoted $x_o = (S, I_T)$.
The endemic equilibrium for bTB when brucellosis is absent is,  $E^{*} = (S_{T}^{*}, I_{T}^{*})$, where $S_{T}^{*} = K \frac{1}{R_{o}^{T}}$ and 
\begin{equation*}
I_{T}^*=\frac{K}{2}\frac{1}{b_1}\left[ 
\left(\frac{bb_1 - \mu - \alpha_T}{r}  - \frac{1+b_1}{\mathcal{R}_0^{T}}\right) + \sqrt{\left(\frac{bb_1 - \mu - \alpha_T}{r}  - \frac{1+b_1}{\mathcal{R}_0^{T}}\right)^2 + 4 \frac{b_1 }{\mathcal{R}_0^{T}}\left(1-\frac{1}{\mathcal{R}_0^{T}}\right)}\right].
\end{equation*}

Now consider the brucellosis only model: 
\begin{align*}
\frac{dS}{dt} &= b (S+ b_2 I_{B} + b_3 R_{B}) \big(1 - \frac{r}{b} \frac{(S+ I_{B} + R_{B})}{K}\big)- \beta_B I_B S - \mu_{S} S  \\		      					% Susceptibles 
\frac{dI_{B}}{dt}&=  \beta_B I_B S - \gamma I_{B} + \epsilon R_{B}  - \mu_{B} I_{B} \\ 		 					% Brucellosis
\frac{dR_{B}}{dt}&=  \gamma I_{B} - \epsilon R_{B} - \mu_{RB} R_{B} \\  										% Chronic brucellosis 
\end{align*}

$R_{o} ^B$ for brucellosis in the case that bTB is absent is, $R_{o}^{B} = \frac{\beta_B K (\epsilon + \mu_R)}{(\gamma + \mu_B) (\epsilon + \mu_{RB}) + \epsilon \gamma}$ \\
The endemic equilibrium for brucellosis when bTB is absent is $E* = (S_{t}^*, I_{Bt}^*, R_{Bt}^*)$, where 
$S_{t}^* = \frac{K}{R_{o} ^B}$, \\
$I_{Bt}^* = \frac{K}{2} \frac{1}{b_2(1+\delta)+ b_3 (\delta + \delta^2)} + \Big[b b_2 + b b_3 \delta - \frac{ \beta_B K}{R_o} - \frac{r + r \delta + b_2 + b_3 \delta}{R_o}\Big] +\\
 \sqrt{4 + \frac{K}{b_2 (1-\delta) + b_3 (\delta + \delta^2)} \Big[b b_2 + b b_3 \delta - \frac{ \beta_B K}{R_o} - \frac{r + r \delta + b_2 + b_3 \delta}{R_o}\Big] ^2}$,\\
$R_{Bt}^* =  \frac{\gamma I_{Bt}^*}{\epsilon - \mu_R}$. \\
$R_o$ for bTB when brucellosis is present is evaluated at these conditions, where $\delta = \frac{\gamma}{\epsilon - \mu_R}.$
$R_{o}^{T} = 2$. \\
CHECK THIS WITH NUMERICAL INTEGRATION!!


\pagebreak 
\noindent
\textbf{Notes on deriving $R_{o}^{T}$}: \\
 $1.$ Note there are $n = 6$ components of this model, $m = 3$ of which are infected with bTB.  Define the vector, $x_o = [I_T, I_C, R_C, S, I_B, R_B]^T$, such that $x_i$ contains the huber of individuals in each component. \\
 $2. $  The rate of change of $x_i$ (e.g. the differential equations) can be represented as, $\mathcal{F}_i(x) - \mathcal{V}_i (x)$.  
 Here $\mathcal{F}_i (x)$ is the rate of appearance of new infections in compartment i and $\mathcal{V}_i (x)$ is the rate of transfer by all other means. 
Noting the negative sign with $\mathcal{V}_i (x)$ and that $\mathcal{F}_i (x)$ only includes infections that are newly arising and does not include terms which describe the transfer of infectious individuals from one infected compartment to another. \\
 Here, $\mathcal{F}_i (x) = \Big[\frac{\beta_T S (I_t + I_C+ R_C)}{S + I_T + I_B + I_C + R_B + R_C}$, $\frac{\beta_{T} I_B (I_t + I_C+ R_C)}{S + I_T + I_B + I_C + R_B + R_C}$, $\frac{\beta_{T} R_B (I_t + I_C+ R_C)}{S + I_T + I_B + I_C + R_B + R_C}, 0, 0, 0\Big]^T$. \\
  $\mathcal{V}_i (x) =\Big[ 2 \Big]^T$ \\
$3.$ Check $\mathcal{F}_i$ and $\mathcal{V}_i$ meet the conditions outlined by van den Driessche and Watmough (2002) so you can use the next generation matrix,  $FV^{-1}$, from m by m matrices of partial derivatives of $F_i$ and $V_i$.   (CHECK ME)
$F_i = \Big[\frac{\partial \mathcal{F}_i (x_o)}{\partial x_i} \Big]$.  $V_i = \Big[\frac{\partial \mathcal{V}_i (x_o)}{\partial x_i} \Big]$.  $x_o$ are the disease free equilibrium values of x.\\
$4.$ Define bTB free equilibrium values $x_o = [I_T = 0, I_C = 0, R_C = 0, S = S*, I_B= I_B*, R_B = R_B*]^T$.  Where $S*$, $I_B*$, and $R_B*$ are the equilibrium values of brucellosis in the absence of bTB.  \\
$5.$ Evaluate the F and V matrices at $x_o$.  \\
Here, $F_i = \begin{bmatrix} \frac{\beta_T S*}{(S* + I_B* + R_B*)} & \frac{\beta_T S*}{(S* + I_B* + R_B*)} & \frac{\beta_T S*}{(S* + I_B* + R_B*)} \\ 
\frac{\beta'_T I_B*}{(S* + I_B* + R_B*)}  & \frac{\beta'_T I_B*}{(S* + I_B* + R_B*)} & \frac{\beta'_T I_B*}{(S* + I_B* + R_B*)} \\
\frac{ \beta'_T R_B*}{(S* + I_B* + R_B*)} & \frac{ \beta'_T R_B*}{(S* + I_B* + R_B*)} & \frac{ \beta'_T R_B*}{(S* + I_B* + R_B*)} \end{bmatrix}$ \\
Here, $V_i = \begin{bmatrix} a&b&c \\ d&e&f \\ g&h&i \end{bmatrix}$ \\
$6.$ Calculate $FV^{-1}$ \\
$7.$ The dominant eigenvalue of $FV^{-1}$ is xxx. \\



\section*{Approach}
Mortality and Fecundity parameters: \\
- Data analyses informs mortality rates for each disease category in \textit{young females aged 2-8)}.  \\
- Data analysis informs fecundity rates for each disease category in \textit{young females aged 2-8)}.  \\
- How to extrapolate to males and other age groups (Figure 3)?  \\
Parameters relevant to the time course of brucellosis are limited to cattle and U.S. bison: \\
- Pull $\gamma$ and $\epsilon$ from the literature.  \\
- Few animals were followed less than 2 years so assume our estimates are relevant for $I_B$.  Still thinking about this. \\
Transmission: \\
- Data analyses inform the proportional increase in transmission with co-infection (e.g. that $\beta'_B = 2.8 * \beta_B$).
 I hope to estimate $\beta_B$ and $\beta_T$ from the time sequence data.

\pagebreak



Let's consider first the TB-only model:
\begin{align}
\frac{dS}{dt}&=b N_b\left(1-\frac{r}{b}\frac{N}{K}\right) -\beta_T S I_T - \mu S\\
\frac{dI_T}{dt}&=\beta_T S I_T - \alpha_T I_T - \mu I_T
\end{align}

Then the disease free equilibrium is $E^0=(K,0)$ where the population is $x=(S,I_T)$. The basic reproduction number is $\mathcal{R}_0=\frac{\beta_T K}{\alpha_T + \mu}$. The endemic equilibrium is $E^*=(S^*,I_T^*)$ where $S^*=K\frac{1}{\mathcal{R}_0}$ and 
\begin{equation*}
I^*=K\left( \frac{r-\alpha_T}{2r}-\frac{1}{\mathcal{R}_0} + \frac{1}{2r}\sqrt{(r-\alpha_T)^2 + \frac{4r\alpha_T}{\mathcal{R}_0}}\right).
\end{equation*}
When $\alpha_T=0$, then $I^*=K(1-1/\mathcal{R}_0)$. We fix other parameters and vary $\beta_T$ to get endemic levels such that $I^*/N=10\%$, etc.
\pagebreak



%In document!
%\usepackage(amsmath)
%\begin{cases}
%r(a, N) &= 
%R(a, N)^\intercal N(a) & \text{if $a = 1$},
%0 & \text{otherwise.},
%\end{cases} \\



\section*{Citations}

Alexander, B., Schnurrenberger, P.R., Brown, R.R. 1981. Numbers of Brucella abortus in the placenta, umbilicus and fetal fluid of two naturally infected cows. Veterinary Record. 108, 500. 

Begon et al. 2002. A clarification of transmission terms in host-microparasite models: numbers, densities and areas. Epidemiol. Infect. 129, 147-153.

Capparelli, R., Parlato, M., Iannaccone, M., Roperto, S., Marabelli, R., Roperto, F., Iannelli, D. 2009. Heterogeneous shedding of Brucella abortus in milk and its effect on the control of animal brucellosis. Journal of Applied Microbiology. 106, 2041-2047.

Cross et al. 2005. Disentangling association patterns in fission?fusion societies using African buffalo as an example. Animal Behaviour. 69, 499-506.

Davis et al. 1990. Brucella abortus in captive bison. I. Serology, bacteriology, pathogenesis, and transmission to cattle. 360-371.

Dolan, L.A. 1980. Latent carriers of brucellosis. 106, 241-243. 

Emminger, A.C., Schalm, O.W. 1943. The effect of Brucella abortus on the bovine udder and its secretion. Am.J. Vet. Res. 4, 100-109, 

Fensterbank, R. 1978. Congenital brucellosis in cattle associated with localization in a hygroma. Veterinary Record. 103, 283-284.

Fuller et al. 2007. Reproduction and survival of Yellowstone bison. JWD. 71, 2365-2372.

McCallum et al. 2001. How should pathogen transmission be modeled. Trends Ecol. Evol. 16, 295-300.

Olsen, S. and Tatum, F. 2010. Bovine Brucellosis.

Plommet et al. 1973. Annales de Recherches Veterinaires. 4, 419. %!!!

Ray, W.C., Brown, R.R., Stringfellow, D.A., Schnurrenberger, P.R., Scanlan, C.M., Swann, A.I. 1988. Bovine brucellosis: an investigation of latency in progeny of culture positive cows. JAVMA. 192, 182-186.

Rhyan, J.C. et al. 2009. Pathogenesis and epidemiology of Brucellosis in Yellowstone bison: Serologic and culture results from adult females and their progeny. J.WD. 45, 729-478.

Rhyan, J. C., W. J. Quinn, L. S. Stackhouse, J. J. Henderson, S. R. Ewalt, J. B. Payeur, M. Johnson, and M. Meagher. 1994. Abortion caused by Brucella abortus biovar 1 in a free- ranging bison (Bison bison) from Yellowstone National Park. Journal of Wildlife Diseases 30:445-446.

Samartino, L.E. and Enright, F.M. 1993. Pathogenesis of abortion of bovine brucellosis. 

Treanor et al. 2010. Vaccination strategies for managing brucellosis in Yellowstone bison. 28S, F64-F72.

Xavier et al. 2009. Pathological, Immunohistochemical, and Bacteriological study of tissues and milk of cows and fetuses experimentally infected with Brucella abortus. J. Comp. Path. 140, 149-157. 

\end{document}